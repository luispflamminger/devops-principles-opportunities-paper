\section{Einleitung}

\subsection{Problemstellung}

Das Geschäftliche Umfeld moderner Unternehmen ist wechselhaft und komplex.
Kundenanforderungen sowie Marktbedingungen verändern sich stetig und der Hohe Grad an
Vernetzung zwischen verschiedenen Anwendungen erschwert die Übersicht.
Aufgrund dieser schweren Bedingungen, benötigen große Softwareunternehmen wie
Microsoft oder Alphabet Möglichkeiten, schnell auf sich ändernde Bedingungen zu reagieren,
Kundenanforderungen rasant umzusetzen und die Zusammenarbeit und Wissensverteilung effizient
zu organisieren. In dieser Arbeit wird untersucht, ob das Konzept DevOps plausible
Chancen bietet, um diesen hohen Anforderungen gerecht zu werden.

\subsection{Zielsetzung}

Ziel dieser Arbeit ist es, die Prinzipien des DevOps Konzepts im Software Engineering
zu beschreiben, sowie seine Chancen für Unternehmen und Entwicklungsteams darzulegen.
Zudem soll DevSecOps thematisch eingeordnet werden. 

\subsection{Aufbau der Arbeit}

Die Arbeit beginnt mit einer Darstellung und Erklärung des Software Lebenszyklus und seiner
Schritte. Anschließend wird das Wasserfallmodell als Beispiel für ein konventionelles Ablaufmodell
erläutert. Agile Entwicklung wird abgegrenzt und anhand von Scrum erklärt.
Im Hauptteil wird zunächst der Begriff DevOps definiert und Ursprung und Ziel des Konzepts
erläutert. Anschließend werden die Grundsätze von DevOps anhand der sechs Säulen dargelegt.
Nachdem die Prinzipien von DevOps klar sind, werden Chancen gegeben, die DevOps modernen
Softwareunternehmen bietet. Daraufhin wird der Begriff DevSecOps im Kontext von DevOps eingeordnet.
Abschließend werden die Erkenntnisse zusammengefasst und mögliche Kritik am Konzept adressiert.