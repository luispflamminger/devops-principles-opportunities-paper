\section{Zusammenfassung und Fazit}

Die Ergebnisse dieser Arbeit zeigen, dass DevOps verschiedene
Methoden, Prozesse, Tools und Denkweisen in sich vereint, um
die Geschwindigkeit des Softwareentwicklungsprozesses 
und der Bereitstellung von Software zu erhöhen.
Cross-funktionale Teams, eine collaborative Kultur,
ein hoher Automatisierungsgrad, KPIs zur Erfolgsmessung und
Arbeitsvisualisierung sowie das Teilen neuer Erkenntnisse sind
Kernbestandteile von DevOps.
Diese Prinzipien bieten viele Chancen für Unternehmen, wie
schnellere Reaktionen auf Kundenanforderungen oder Veränderungen
am Markt, Reduktion von Komplexität und Unsicherheit oder
bessere Übersicht und Entscheidungsfindung durch effizientere
Zusammenarbeit. Zum Schluss wurde DevSecOps erläutert und eingeordnet.

Zusätzlich zu den Chancen, die in dieser Arbeit dargestellt wurden
gibt es aber auch Kritik am DevOps Konzept.
So gibt es Stimmen die behaupten, dass DevOps nicht funktioniere,
da es in einer überwiegenden Anzahl an Unternehmen nicht die gewünschten
Ergebnisse bringe \cite{Halstenberg2020}.
Eine andere These ist, dass DevOps sich in vielen Bereichen nicht mit der
Regulatorik vereinbaren lässt, z.B. im Bankensektor, in dem feste Rahmenbedingungen
für die Durchführung und Dokumentation von Softwareprojekten existieren.
Es wird zudem behauptet, dass DevOps nicht in der Lage ist, Silos aufzubrechen, sondern
diese lediglich verschiebt \cite{Halstenberg2020}.

Trotz dieser Kritikpunkte wird DevOps in den größten Softwareunternehmen eingesetzt
und sorgt dafür, dass die Entwicklung mit der schnellebigen modernen Welt mithalten
kann. Jede IT-Abteilung sollte Prüfen, ob sich durch den Einsatz von DevOps Prinzipien
ein wirtschaftlicher und organisatorischer Vorteil erreichen lässt.


