\newpage
\section{Prinzipien und Chancen von DevOps}

\subsection{Definition und Ziel von DevOps}

Der Begriff DevOps setzt sich aus den Begriffen Development und 
Operations (engl. Betrieb) abgeleitet ist \cite{Alt2017}.
Development steht dabei für die Softwareentwicklung,
während Operations für den IT-Betrieb steht \cite{DevOpsAgileHeroes}.

DevOps ist eine Kombination von kulturellen und organisatorischen Denkweisen,
Prozessen und Tools, die darauf ausgerichtet sind, den Zeitraum zwischen 
der Planung und Entwicklung einer Änderung an der Software,
bis zur Bereitstellung dieser
Änderung für Kunden auf dem Produktivsystem möglichst weit zu verringern \cite{AmazonDevOps}.
Eine Änderung kann dabei ein neues Produkt, Feature oder das Beheben eines Fehlers sein.
Durch diese erhöhte Geschwindigkeit können Unternehmen die Time-to-Market,
also die Zeit, bis ein neues Feature oder Produkt den Markt erreicht,
drastisch reduzieren, was es ihnen in erster Linie erlaubt,
schnell auf Feedback der Kunden und Veränderungen auf dem Markt reagieren zu können,
um einen Wettbewerbsvorteil zu erlangen. Auch Fehler können schneller angegangen werden,
was eine kontinuierliche Verbesserung des Service zur Folge hat \cite{Halstenberg2020}.

DevOps nahm seinen Ursprung bei einer Konferenz in 2009, bei der das Team von Flickr
einen Prozess vorstellte, mit dem es in der Lage ist, neue Änderungen an der Software
bis zu zehn mal am Tag in das Produktionssystem zu übernehmen.
Inspiriert ist die Bewegung auch durch das \ac{TPS}, welches durch
T. Ohno bei Toyota entwickelt wurde \cite{Halstenberg2020}.
Es ist ein Produktionssystem in der Autoindustrie, das einzig darauf ausgerichtet ist,
den Moment zwischen dem Kundenauftrag und dem Geldeingang zu minimieren.
Konzepte aus \ac{TPS} wurden auf den IT-Wertschöpfungsprozess umgelegt,
um von den gleichen Vorteilen profitieren zu können \cite{Halstenberg2020}.

DevOps stellt keinen festgelegten Standard, wie z.B. SCRUM dar,
sondern entwickelt sich kontinuierlich weiter.
Es beruht auf bereits etablierten Konzepten wie Agilität, IT-Service Management,
KanBan und Automatisierung. Dabei erfindet es diese Konzepte nicht neu, sondern bietet
die Möglichkeit, sie zu nutzen, um eine Brücke zwischen den drei Seiten 
Entwicklung, Betrieb und Business zu schaffen \cite{Halstenberg2020}

\subsection{Die Säulen von DevOps}

Das Akronym \ac{CAMS}, oder erweitert \ac{CALMAS}, wird verwendet, um
Einige Grundsätze von DevOps zusammenzufassen.

\textbf{Culture} (Kultur):

Die Kultur im Unternehmen und im Team ist ein wichtiger Bestandteil von DevOps.
Grundlage der Kultur ist die intensive Zusammenarbeit miteinander.
Die Gewinnung und das Verteilen von Informationen wird gefördert. Das Überbringen
schlechter Nachrichten ist wichtig und wird befürwortet, um Fehler schnell beheben
zu können. Dabei werden diese nicht verurteilt oder bestraft, sondern untersucht und
korrigiert. Risiken werden geteilt und abteilungsübergreifendes Handeln wird gefördert.
Ein wichtiger Teil der Kultur ist es zudem, ständig neue Ansätze auszuprobieren,
um sich kontinuierlich weiterzuentwickeln \cite{Halstenberg2020}.

\textbf{Automation} (Automatisierung):

Die gesamte Kette des Softwareentwicklungsprozesses sollte automatisiert abgebildet werden.
Dies wird über sog. \ac{CI/CD} Pipelines realisiert und umfasst unter anderem 
automatisierte Tests, Bereitstellung und Produktivnahme der Software.
Durch Automatisierung können Liegezeiten und Fehler vermieden, sowie
Entlastung bei den Mitarbeitern geschaffen werden \cite{Alt2017}\cite{Halstenberg2020}.

\textbf{Lean}:

Lean-Methoden umfassen vor allem leichtgewichtige Entscheidungsprozesse, kontinuierliche
Messungen und die Visualisierung von Arbeit.
Auch die Limitierung von Work-in-Progress Items ist wichtig um Durchlaufzeiten zu verringern.
Sie begrenzen die Anzahl an Aufgaben, die ein Mitarbeiter oder das gesamte Team
gleichzeitig in Bearbeitung haben darf \cite{Halstenberg2020}.

\textbf{Measurement} (Messung):

Um die Produktivität des DevOps Prozesses steigern zu können, sind \acp{KPI} nötig,
um die Leistung des Teams einstufen und messbare Ziele definieren zu können.
Interessante Kennzahlen sind dabei z.B. Verfügbarkeiten, die Dauer vom Auftreten
eines Fehlers bis zur Behebung oder die Zeit von einer Änderung im Code, bis diese in
der Produktion zu sehen ist.
Zudem sind Kennzahlen, die den geschäftlichen Nutzen der IT messen wichtig \cite{Halstenberg2020}.

\textbf{Added Value} (Mehrwert):

DevOps darf nur eingesetzt werden, wenn es einen konkreten Mehrwert für das Unternehmen
bringt. Ein gutes Beispiel für solch einen Mehrwert, ist die schnellere Auslieferung
eines Features an einen Kunden oder das schnellere Beheben eines Fehlers \cite{Halstenberg2020}.
Kann kein messbarer Mehrwert festgestellt werden, wurde DevOps nicht erfolgreich implementiert.


\textbf{Sharing} (Teilen):

Eng mit dem Punkt "Culture" ist auch der Punkt "Sharing" verbunden.
Er beschreibt das Prinzip des Teilens von für die Zusammenarbeit notwendigem Wissen
mit allen Beteiligten Mitarbeitern. Dieses Wissen kann die das Produkt selbst,
aber auch Prozesse und Tools, die den DevOps Prozess erleichtern, umfassen.
Wie bereits genannt ist das Teilen von Fehlern auch ein zentraler Bestandteil und
muss ohne Schuldzuweisungen, sondern Lösungsorientiert, erfolgen \cite{Halstenberg2020}.

\subsection{Chancen von DevOps}

Die heutige Geschäftsumgebung lässt sich gut durch das Akronym \ac{VUCA} darstellen,
welches Zusammengefasst folgende Punkte beinhaltet.
Die Märkte und Kundenanforderungen sind wechselhaft und verändern sich ständig.
Dadurch entsteht eine gewisse Unsicherheit, die es erschwert, langfristige Projekte
und Strategien zu planen \cite{Halstenberg2020}.
Mit dem hohen Grad an Vernetzung zwischen verschiedenen Anwendungen und Prozessen
geht eine hohe Komplexität einher. Der Zusammenhang zwischen Ursache und Wirkung
ist oft unklar. Zudem sind meist verschiedene Handlungsalternativen als Antwort
auf bestimmte Informationen möglich, da diese durch komplexe Zusammenhänge
oft mehrdeutig interpretiert werden können \cite{Halstenberg2020}.

DevOps bietet eine Antwort auf fast alle dieser Punkte und bietet somit die Chance,
in einer VUCA-Welt geschäfts- und handlungsfähig zu bleiben.

Durch die hohe Geschwindigkeit ist DevOps in der Lage, die Time-to-Market erheblich
zu senken. In einer VUCA-Welt hat dies mehrere Vorteile.
Zum einen kann wesentlich schneller auf Veränderungen im Markt reagiert werden,
da die Durchlaufzeit von Anpassungen am Produkt oder neuen Features geringer ist.
Dies erhöht die Chance, dass Kundenanforderungen nicht bereits veraltet sind,
wenn die Änderungen fertiggestellt werden \cite{Halstenberg2020}

Durch die rasche Bereitstellung verringert sich auch die Time-to-Value, das heißt die 
Zeit, bis der Kunde einen ersten Mehrwert aus einem Produkt ziehen kann.
Dies wird durch das kontinierliche Deployment neuer Änderungen erreicht. So erhält
der Kunde zu Anfang zwar kein fertiges Produkt, kann aber bereits Vorteile aus einem
Prototypen ziehen \cite{Halstenberg2020}. Zudem verringert sich so die Zeit, bis Feedback eingeholt werden kann
was Unsicherheit und Komplexität verringert und Planungssicherheit gibt.

Die verbesserte Zusammenarbeit innerhalb cross-funktionaler Teams mit guter Kultur
kann helfen, die Komplexität und Ambiguität des Geschäftsumfelds zu verringern.
Bessere Kommunikation und unterschiedliche Skillsets ermöglichen einen besseren Überblick
über verschiedene Handlungsoptionen und erleichern die Entscheidungsfindung.

Abgesehen von VUCA bietet DevOps weitere Chancen.
Die Zuverlässigkeit und Qualität kann durch automatisiertes Testing und Bereitstellung
erhöht werden. Durch automatisierte Prozesse wird die Fehleranfälligkeit
reduziert und die allgemeine Codequalität steigt \cite{AmazonDevOps}.

Durch automatisierte Prozesse ist sichergestellt, dass auch Projekte mit großem 
Umfang zügig bereitgestellt werden können. Dies ist möglich, da Automatisierungen und die 
unterliegende virtuelle Infrastruktur sehr gut skalieren können \cite{AmazonDevOps}.

Eine weitere Chance, die DevOps bietet, ist erhöhte Sicherheit.
Diese wird vor allem im erweiterten Modell DevSecOps mit in Betracht gezogen,
welches nachfolgend eingeordnet wird.

\subsection{Einordnung von DevSecOps}

DevSecOps erweitert den DevOps Gedanken um ein weiteres Themenfeld, die Sicherheit oder
Security. In diesem Ansatz ist die Software Sicherheit ein integraler Bestandteil
des IT-Lifecycles \cite{RedHatDevSecOps}.
Traditionell ist IT-Security eine abgetrennte, der Softwareentwicklung nachgelagerte Funktion
und Aufgabe eines seperaten Teams.
Mit schnellen Entwicklungszyklen, wie sie DevOps bietet, ist das Risiko groß,
dass die Sicherheit den Entwicklungs- und Bereitstellungsprozess ausbremst, da
sie nicht schnell genug auf Veränderungen reagieren kann \cite{RedHatDevSecOps}.

DevSecOps berücksichtigt die Anwendungssicherheit von Anfang an. Dies beinhaltet die
Automatisierung von sicherheitsrelevanten Tests und Prozessen,
um den DevOps Prozess nicht zu verlangsamen und Releasezyklen schnell zu halten.
Neben zusätzlicher Automatisierung und Tools ist aber auch ein Kulturwechsel nötig.
Sicherheitsteams müssen von Beginn an fest mit in den Prozess integriert werden.
Entwickler müssen offen mit Sicherheitsexperten kommunizieren und Bedrohungen müssen
offen Kommuniziert werden, ähnlich der Fehlerkultur im regulären DevOps \cite{RedHatDevSecOps}.