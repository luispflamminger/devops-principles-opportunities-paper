% Optionen für Biblatex
\ExecuteBibliographyOptions{%
giveninits=false,
isbn=true,
url=true,
doi=false,
eprint=false,
maxbibnames=7, % Alle Autoren (kein et al.)
maxcitenames=2, % et al. ab dem 3. Autor
backref=false, % Rückverweise auf Zitatseiten
bibencoding=utf8, % wenn .bib in utf8, sonst ascii
bibwarn=true, % Warnung bei fehlerhafter bib-Datei
}%

% et al. an Stelle von u.a.
\DefineBibliographyStrings{ngerman}{
   andothers = {{et\,al\adddot}},
}

% Klammern um das Jahr in der Fußnote
\renewbibmacro*{cite:labelyear+extrayear}{%
  \iffieldundef{labelyear}
    {}
    {\printtext[bibhyperref]{%
       \mkbibparens{%
         \printfield{labelyear}%
         \printfield{extrayear}}}}}

\renewbibmacro*{cite:title}{%
  \printtext[bibhyperref]{%
    \printfield[citetitle]{labeltitle}%
    \setunit{\addcomma\space}%
    \printdate}}

\DeclareNameFormat{last-first}{%
  \iffirstinits
    {\usebibmacro{name:family-given}
        {\namepartfamily}
        {\namepartgiveni}
        {\namepartprefix}
        {\namepartsuffix}
    }
    {\usebibmacro{name:family-given}
        {\namepartfamily}
        {\namepartgiven}
        {\namepartprefix}
        {\namepartsuffix}
    }%
  \usebibmacro{name:andothers}}

% Alternative Notation der Fußnoten
% Zeigt sowohl den Nachnamen als auch den Vornamen an
% Beispiel: \fullfootcite[Vgl. ][Seite 5]{Tanenbaum.2003}
\DeclareCiteCommand{\fullfootcite}[\mkbibfootnote]
  {\usebibmacro{prenote}}
  {\usebibmacro{citeindex}%
    \printnames[sortname][1-1]{author}%
    \addspace (\printfield{year})}
  {\addsemicolon\space}
  {\usebibmacro{postnote}}

%Autoren (Nachname, Vorname)
\DeclareNameAlias{default}{family-given}

%Reihenfolge von publisher, year, address verändern
% Achtung, bisher nur für den Typ @book definiert

%% Definiert @Book Eintrag
\DeclareBibliographyDriver{book}{%
  \printnames{author}%
  \newunit\addcolon\space
  \printfield{title}%
  \setunit*{,\space}%
  \printfield{edition}%
  \setunit*{\addcomma\space}%
  \printlist{publisher}%
  \newunit\newblockpunct
  \printlist{location}%
  \setunit*{\space}%
  \printfield{year}%
  \setunit*{,\space}%
  \printfield{isbn}%
  \finentry}

%% Definiert @Online Eintrag
\DeclareBibliographyDriver{online}{%
  \printnames{author}%
  \newunit\newblockpunct
  \printfield{title}%
  \setunit*{,\space}%
  %\newunit\newblock
  \printfield{url}%
  \setunit*{,\space Erscheinungsjahr:\space}%
  \printfield{year}%
  \setunit*{,\space Aufruf am:\space}%
  \printfield{note}%
  \finentry}

%% Definiert @Article Eintrag
\DeclareBibliographyDriver{article}{%
  \printnames{author}%
  \newunit\newblockpunct
  \printfield{title}%
  \setunit*{.\space In:\space}%
  %\newunit\newblock
  \usebibmacro{journal}%
  \setunit*{\space (}%
  \printfield{year}\newunit{)}%
  \finentry}

%% Definiert @InProceedings Eintrag
\DeclareBibliographyDriver{inproceedings}{%
	\printnames{author}%
	\setunit*{,\space (}%
	\printfield{year}\newunit{)}%
	\newunit\newblockpunct
	\printfield{title}%
	\setunit*{\space}%
	\usebibmacro{booktitle}%
	\setunit*{,\space}%
	\printfield{isbn}%
	\setunit*{,\space}%
	\printfield{doi}%
	\finentry}

%Doppelpunkt nach dem letzten Autor
\renewcommand*{\labelnamepunct}{\addcolon\addspace }

%Komma an Stelle des Punktes
\renewcommand*{\newunitpunct}{\addcomma\space}

%Autoren durch Semikolon trennen
\newcommand*{\bibmultinamedelim}{\addsemicolon\space}%
\newcommand*{\bibfinalnamedelim}{\addsemicolon\space}%
\AtBeginBibliography{%
  \let\multinamedelim\bibmultinamedelim
  \let\finalnamedelim\bibfinalnamedelim
}

%Titel nicht kursiv anzeigen
\DeclareFieldFormat{title}{#1\isdot}
